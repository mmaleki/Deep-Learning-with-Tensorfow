%\documentclass{beamer}
\documentclass[aspectratio=13]{beamer}

\usepackage{graphicx}
\usetheme{Warsaw}

\usepackage[utf8]{inputenc}


%Information to be included in the title page:
\title{\huge{Twitter Sentiment Analysis}}
\author{\text{Dr. Mehrdad Maleki}}


\date{}

\begin{document}

\frame{\titlepage}

\begin{frame}
\frametitle{What is Sentiment Analysis?}
\begin{itemize}
\setlength\itemsep{1em}
\item 
\end{itemize}
\end{frame}


\begin{frame}
\frametitle{What we get by the end of this short course?}
\begin{itemize}
\setlength\itemsep{1em}
\item 
\end{itemize}
\end{frame}


\begin{frame}
\frametitle{Our dataset}
This is the sentiment140 dataset. It contains 1,600,000 tweets extracted using the twitter api . The tweets have been annotated ($0 = negative$, $4 = positive$) and they can be used to detect sentiment. It contains the following 6 fields:
\begin{itemize}
\setlength\itemsep{1em}
\item \textbf{target}: the polarity of the tweet ($0 = negative$, $2 = neutral$, $4 = positive$)
\item \textbf{ids}: The id of the tweet ( 2087)
%\item date: the date of the tweet (Sat May 16 23:58:44 UTC 2009)
\item \textbf{flag}: The query (lyx). If there is no query, then this value is NO\_QUERY.
\item \textbf{user}: the user that tweeted (robotickilldozr)
\item \textbf{text}: the text of the tweet (Lyx is cool)
\end{itemize}
\end{frame}



\begin{frame}
\begin{figure}
\includegraphics[scale=0.20]{scrshot.png}
\end{figure}
\end{frame}


\begin{frame}
\frametitle{Reading data}
\begin{itemize}
\setlength\itemsep{1em}
\item 
\end{itemize}
\end{frame}




\end{document}