\documentclass[12pt,aspectratio=169]{beamer}


\usepackage{algorithm,algorithmic}

\usepackage[utf8]{inputenc}
\usepackage{booktabs}
\usepackage[opacity=0.1]{pdfcomment} % set to 0 to make annotation icons invisible
\usepackage{pdfpc}
\usepackage{arev}
\usepackage{multicol}

\usepackage{xcolor, color, colortbl}
\definecolor{dkgreen}{rgb}{0,0.5,0}
\definecolor{dkred}{rgb}{0.8,0,0}
\definecolor{dkblue}{rgb}{0,0,0.5}
\definecolor{gray}{rgb}{0.5,0.5,0.5}
\definecolor{mauve}{rgb}{0.58,0,0.82}
\definecolor{hilight}{RGB}{122,86,0}

\definecolor{LRed}{rgb}{1,.8,.8}
\definecolor{MRed}{rgb}{1,.6,.6}
\definecolor{HRed}{rgb}{1,.2,.2}

\usepackage{tikz}
\usetikzlibrary{arrows.meta,
                calc, chains,
                quotes,
                positioning,
                shapes.geometric}


\def\scalefact{0.85}
\newcommand{\cev}[1]{\reflectbox{\ensuremath{\vec{\reflectbox{\ensuremath{#1}}}}}}
\newcommand{\evalat}[2]{\left.#1\right\vert_{#2}}

\newcommand{\znode}[5][black]{\path (#3,#4) node(#2) [circle,draw,color=#1] {#5};}
\newcommand{\zunedge}[6][black]{%
\begin{scope}
	\path (#2,#3) node(this) [inner sep=0pt,triangle,draw,color=#1] {#4};
	\draw[->,color=#1] (#5) -- (this.west);
	\draw[->,color=#1] (this.east) -- (#6);
\end{scope}}
\newcommand{\zbiedge}[7][black]{%
\begin{scope}
	\path (#2,#3) node(this) [inner sep=0pt,triangle,draw,color=#1] {#4};
	\draw[->,color=#1] (#5) -- (this);
	\draw[->,color=#1] (#6) -- (this);
	\draw[->,color=#1] (this.east) -- (#7);
\end{scope}}
\newcommand{\zedge}[5][black]{\path (#3,#4) node(#2) [inner sep=0pt,triangle,draw,color=#1] {#5};}

\definecolor{blue(pigment)}{rgb}{0.2, 0.2, 0.6}
\definecolor{burgundy}{rgb}{0.5, 0.0, 0.13}


\usepackage{listings}
%% \usetheme{Goettingen}
\usefonttheme{serif}
\usepackage{times}
\setbeamertemplate{navigation symbols}{}

\title{Deep Learning}
\subtitle{Lecture 7: Backpropagation}
 
 
%\author[Mehrdad Maleki] % (optional, for multiple authors)
%{Mehrdad Maleki, Barak A. Pearlmutter\footnote{ Institute & Department of Computer Science
%Maynooth University, Co. Kildare, Ireland}, Jeffrey Mark Siskind}
 
%\institute[NUIM] % (optional)
%{
%  Department of Computer Science \\
%  National University of Ireland Maynooth
 
%}

\author[]{\textbf{Dr. Mehrdad Maleki}}
%\institute[]{\textsuperscript{1}Department of Computer Science\\ National University of Ireland\\ Maynooth}
 
\date{}
 
%\logo{\includegraphics[height=1.5cm]{lion-logo.png}}

\renewcommand{\Re}{\mathbb{R}}
 
\begin{document}
 
\frame{\titlepage}

\begin{frame}
\frametitle{Notations}
\begin{center}
\begin{tikzpicture}
\foreach \x in {0,4}
  \foreach \y in {-2,-1,0,1,2}
    \draw[ultra thick, blue!50] (\x,\y) circle (10pt);

  \foreach \y in {-2,-1,0,1,2}
       \draw[->] (0.4,\y) -- (3.65,0);
  \draw[->] (4.35,0) -- (5,0);\pause

\node[red!70] at (0,3) {$l-1$};
\node[red!70] at (4,3) {$l$};\pause

\node[opacity=0.3] at (0,-3) {$m$};
\node[opacity=0.3] at (4,-3) {$n$};\pause

\node[brown] at (2,2.5) {$W^{(l)}$};\pause

\node[] at (0,0) {$i$};\pause
\node[] at (4,0) {$j$};\pause

\node[brown!50] at (2,0) {$w_{ij}^{(l)}$};\pause

\node[blue!50] at (1,0) {$y_i^{(l-1)}$};\pause

\node[pink] at (3.65,0.3) {$z_j^{(l)}$};\pause

\node[blue!50,right] at (5,0) {$y_j^{(l)}=f_j^{(l)}(z_j^{(l)})$};

\end{tikzpicture}
\end{center}
\end{frame}


\begin{frame}
\begin{itemize}
\item $z_j^{(l)}=\sum_{k=1}^m y_k^{(l-1)}w_{kj}^{(l)}+b_j^{(l)}$\pause
\bigskip
\item $y_j^{(l)}=f_j^{(l)}(z_j^{(l)})$\pause
\bigskip
\item $y_j^{(0)}=x_j$ the $j$-th input.\pause
\bigskip
\item $y_j^{(L)}=$ the $j$-th output.
\end{itemize}
\end{frame}

\begin{frame}
\frametitle{Matrix form}
\begin{itemize}
\item For $l=1,\dots, L$,
\[
\mathbf{z}^{(l)}=
\begin{bmatrix}
z_1^{(l)}\\
\vdots\\
z_{m_l}^{(l)}
\end{bmatrix}
\]\pause
\[
\mathbf{b}^{(l)}=
\begin{bmatrix}
b_1^{(l)}\\
\vdots\\
b_{m_l}^{(l)}
\end{bmatrix}
\]\pause
\item For $l=0,\dots, L$,
\[
\mathbf{y}^{(l)}=
\begin{bmatrix}
y_1^{(l)}\\
\vdots\\
y_{m_l}^{(l)}
\end{bmatrix}
\]
\end{itemize}
where $m_l$ is the number of neurons in layer $l$.
\end{frame}

\begin{frame}
The weight matrix between layer $l-1$ and $l$ which is a $m_{l-1}\times m_l$ matrix as follow,
\[
\mathbf{W}^{(l)}=
\begin{bmatrix}
w_{11}^{(l)} & \dots & w_{1m_l}^{(l)}\\
\vdots & \ddots & \vdots\\
w_{m_{l-1}1}^{(l)} & \dots & w_{m_{l-1}m_l}^{(l)}
\end{bmatrix}
\]

\end{frame}

\begin{frame}
\begin{itemize}
\item $\mathbf{z}^{(l)}=(\mathbf{W}^{(l)})^T\mathbf{y}^{(l-1)}+\mathbf{b}^{(l)}$ \pause
\bigskip
\item $\mathbf{y}^{(l)}=\mathbf{f}^{(l)}(\mathbf{z}^{(l)})$
\bigskip
\end{itemize}
where, \pause
\[
\mathbf{f}^{(l)}=
\begin{bmatrix}
f_1^{(l)}\\
\vdots\\
f_{m_l}^{(l)}
\end{bmatrix}
\]
and\pause
\[
\mathbf{f}^{(l)}(\mathbf{z}^{(l)})=
\begin{bmatrix}
f_1^{(l)}(z_1^{(l)})\\
\vdots\\
f_{m_l}^{(l)}(z_{m_l}^{(l)})
\end{bmatrix}
\] 

\end{frame}

\begin{frame}
\begin{figure}
\begin{tikzpicture}

\foreach \x in {0,2,4}
  \foreach \y in {-2,-1,0,1,2}
    \draw[ultra thick, blue] (\x,\y) circle (10pt);
    
\foreach \y in {-1,1}
   \draw[ultra thick,brown] (-2,\y) circle (10pt);
    
\foreach \y in {-1,0,1}
   \draw[ultra thick,red] (6,\y) circle (10pt);
\foreach \x in {0,2}
  \foreach \y in {-2,-1,0,1,2}
    \foreach \d in {-2,-1,0,1,2}
       \draw[->] (\x+0.4,\y) -- (\x+1.6,\d);
\foreach \y in {-1,1}
    \foreach \d in {-2,-1,0,1,2}
       \draw[->] (-2+0.4,\y) -- (-2+1.6,\d);
\foreach \y in {-1,0,1}
    \foreach \d in {-2,-1,0,1,2}
       \draw[->] (4+0.4,\d) -- (5.6,\y);
\foreach \d in {-1,0,1}
       \draw[->] (6.4,\d) -- (7,\d);
      
\node[ultra thick,red!80] at (1,1.5) {$\mathbf{W_{ij}^{(2)}}$}; 


\draw[fill=green!50] (0,2) circle (10pt);
\draw[fill=red!50] (2,1) circle (10pt); 
\node[] at (0,2) {$i$};
\node[] at (2,1) {$j$};     
\pause
\node[] at (-1,-3) {$W_{ij}^{(1)}$};
\node[] at (1,-3) {$W_{ij}^{(2)}$};
\node[] at (3,-3) {$W_{ij}^{(3)}$};
\node[] at (5,-3) {$W_{ij}^{(4)}$};

\pause

\node[] at (-1,-4) {$b_j^{(1)}$};
\node[] at (1,-4) {$b_j^{(2)}$};
\node[] at (3,-4) {$b_j^{(3)}$};
\node[] at (5,-4) {$b_j^{(4)}$};

\pause

\node[] at (-2,-5) {$y_j^{(0)}$};
\node[] at (0,-5) {$y_j^{(1)}$};
\node[] at (2,-5) {$y_j^{(2)}$};
\node[] at (4,-5) {$y_j^{(3)}$};
\node[] at (6,-5) {$y_j^{(4)}=y_{out}$};

\pause

\draw[rounded corners, ultra thick, fill=pink] (7,-1.2) rectangle (9,1.2);

\foreach \x in {7.2,8, 8.8}
  \draw[->] (\x,2) -- (\x,1.2);
  
\pause 
  
\node[above] at (7.2,2) {$d_1$};
\node[above] at (8,2) {$d_2$};
\node[above] at (8.8,2) {$d_3$};

\node[] at (8,0) {$L^2\;Div$};

\pause

\foreach \d in {0}
       \draw[->] (9,\d) -- (10,\d);
\node[right] at (10,0) {$\mathcal{L}$};
\end{tikzpicture}
\end{figure}
\end{frame}

\begin{frame}
For real-valued output vector, we use $L^2$ norm for divergence which is
\[
Div(\mathbf{Y},\mathbf{d})=\|\mathbf{Y}-\mathbf{d}\|
\] 
For binary classifier with scalar output, $Y\in (0,1)$, the desire output is $0$ or $1$ and the cross entropy between the probability distribution $[Y,1-Y]$ and the ideal output probability $[d,1-d]$ is the \textbf{Kullback–Leibler} divergence, i.e.,
\[
Div(Y,d)=-d\log Y-(1-d)\log (1-Y)
\]
\end{frame}

\begin{frame}
\begin{figure}
\begin{tikzpicture}

\foreach \x in {0,2,4}
  \foreach \y in {-2,-1,0,1,2}
    \draw[ultra thick, blue] (\x,\y) circle (10pt);
    
\foreach \y in {-1,1}
   \draw[ultra thick,brown] (-2,\y) circle (10pt);
    
\foreach \y in {0}
   \draw[ultra thick,red] (6,\y) circle (10pt);
\foreach \x in {0,2}
  \foreach \y in {-2,-1,0,1,2}
    \foreach \d in {-2,-1,0,1,2}
       \draw[->] (\x+0.4,\y) -- (\x+1.6,\d);
\foreach \y in {-1,1}
    \foreach \d in {-2,-1,0,1,2}
       \draw[->] (-2+0.4,\y) -- (-2+1.6,\d);
\foreach \y in {0}
    \foreach \d in {-2,-1,0,1,2}
       \draw[->] (4+0.4,\d) -- (5.6,\y);
\foreach \d in {0}
       \draw[->] (6.4,\d) -- (7,\d);
       
\node[ultra thick,red] at (1,1.5) {$\mathbf{W_{ij}^{(2)}}$}; 
\node[] at (0,2) {$j$};
\node[] at (2,1) {$i$};      

\node[] at (-1,-3) {$W_{ij}^{(1)}$};
\node[] at (1,-3) {$W_{ij}^{(2)}$};
\node[] at (3,-3) {$W_{ij}^{(3)}$};
\node[] at (5,-3) {$W_{ij}^{(5)}$};


\node[] at (-1,-4) {$b_j^{(1)}$};
\node[] at (1,-4) {$b_j^{(2)}$};
\node[] at (3,-4) {$b_j^{(3)}$};
\node[] at (5,-4) {$b_j^{(5)}$};

\node[] at (-2,-5) {$y_j^{(0)}$};
\node[] at (0,-5) {$y_j^{(1)}$};
\node[] at (2,-5) {$y_j^{(2)}$};
\node[] at (4,-5) {$y_j^{(3)}$};
\node[] at (6,-5) {$y_j^{(4)}=y_i$};



\draw[rounded corners, ultra thick, fill=pink] (7,-1.2) rectangle (9,1.2);

\foreach \x in {8 }
  \draw[->] (\x,2) -- (\x,1.2);
  

\node[above] at (8,2) {$d$};


\node[] at (8,0) {$KL\;Div$};

\foreach \d in {0}
       \draw[->] (9,\d) -- (10,\d);
\node[right] at (10,0) {$\mathcal{L}$};
\end{tikzpicture}
\end{figure}
\end{frame}

\begin{frame}
\frametitle{Backpropagation}
The goal of this section is to calculate $\frac{\partial\mathcal{L}}{\partial w_{ij}^{(1)}}$ recursively. 
\[
\begin{aligned}
\frac{\partial \mathcal{L}}{\partial y_j^{(3)}}&=\frac{\partial \mathcal{L}}{\partial y}\frac{\partial y}{\partial y_j^{(4)}}\frac{\partial y_j^{(4)}}{\partial z_j^{(4)}}\frac{\partial z_j^{(4)}}{\partial y_j^{(3)}}
\end{aligned}
\]
If we let $\delta y_j^{(l)}=\frac{\partial y}{\partial y_j^{(l)}}$ then,
\[
\begin{aligned}
\frac{\partial \mathcal{L}}{\partial y_j^{(1)}}&=\frac{\partial \mathcal{L}}{\partial y}\delta y_j^{(1)}
\end{aligned}
\]
\end{frame}


\begin{frame}
\frametitle{Backpropagation}
On the other hand
\[
\begin{aligned}
\frac{\partial \mathcal{L}}{\partial w_{ij}^{(1)}}=\frac{\partial \mathcal{L}}{\partial y}\frac{\partial y}{\partial y_j^{(4)}}&\frac{\partial y_j^{(4)}}{\partial z_j^{(4)}}\frac{\partial z_j^{(4)}}{\partial y_j^{(3)}}\\ \pause
&\frac{\partial y_j^{(3)}}{\partial z_j^{(3)}}\frac{\partial z_j^{(3)}}{\partial y_j^{(2)}}\\ \pause
&\frac{\partial y_j^{(2)}}{\partial z_j^{(2)}}\frac{\partial z_j^{(2)}}{\partial y_j^{(1)}}\\ \pause
&\frac{\partial y_j^{(1)}}{\partial z_j^{(1)}}\frac{\partial z_j^{(1)}}{\partial w_{ij}^{(1)}}
\end{aligned}
\]

\end{frame}


\begin{frame}
\begin{itemize}
\item $\textbf{loss}=\mathcal{L}(\mathbf{y}^{(L)},\mathbf{d};\mathbf{W})$\pause
\item $\mathbf{y}^{(L)}=\mathbf{f}^{(L)}(\mathbf{z}^{(L)})$\pause
\item $\mathbf{z}^{(L)}=(\mathbf{W}^{(L)})^T\mathbf{y}^{(L-1)}+\mathbf{b}^{(L)}$ \pause
\item $\vdots$
\item $\mathbf{y}^{(l)}=\mathbf{f}^{(l)}(\mathbf{z}^{(l)})$\pause
\item $\mathbf{z}^{(l)}=(\mathbf{W}^{(l)})^T\mathbf{y}^{(l-1)}+\mathbf{b}^{(l)}$ \pause
\item $\vdots$
\item $\mathbf{y}^{(1)}=\mathbf{f}^{(1)}(\mathbf{z}^{(1)})$\pause
\item $\mathbf{z}^{(1)}=(\mathbf{W}^{(1)})^T\mathbf{y}^{(0)}+\mathbf{b}^{(1)}$ \pause

\end{itemize}
\end{frame}

\begin{frame}
We want to compute $\frac{\partial \mathcal{L}}{\partial \mathbf{W}^{(l)}}$. But,
\[
\begin{aligned}
\frac{\partial \mathcal{L}}{\partial \mathbf{W}^{(l)}}=\frac{\partial \mathcal{L}}{\partial \mathbf{y}^{(L)}}\frac{\partial \mathbf{y}^{(L)}}{\partial \mathbf{y}^{(l)}}\frac{\partial \mathbf{y}^{(l)}}{\partial \mathbf{z}^{(l)}}\frac{\partial \mathbf{z}^{(l)}}{\partial \mathbf{W}^{(l)}}
\end{aligned}
\]

\end{frame}

\begin{frame}
We can compute $\frac{\partial \mathbf{y}^{(L)}}{\partial \mathbf{y}^{(l)}}$ by the chain rule as follow,
\[
\begin{aligned}
\frac{\partial \mathbf{y}^{(L)}}{\partial \mathbf{y}^{(l)}}=\frac{\partial \mathbf{y}^{(L)}}{\partial \mathbf{z}^{(L)}}\frac{\partial \mathbf{z}^{(L)}}{\partial \mathbf{y}^{(L-1)}}\frac{\partial \mathbf{y}^{(L-1)}}{\partial \mathbf{y}^{(l)}}
\end{aligned}
\]
By getting transpose we have
\[
\begin{aligned}
\frac{\partial \mathcal{L}}{\partial \mathbf{W}^{(l)}}^T=\frac{\partial \mathbf{z}^{(l)}}{\partial \mathbf{W}^{(l)}}^T\frac{\partial \mathbf{y}^{(l)}}{\partial \mathbf{z}^{(l)}}^T\frac{\partial \mathbf{y}^{(L)}}{\partial \mathbf{y}^{(l)}}^T\frac{\partial \mathcal{L}}{\partial \mathbf{y}^{(L)}}^T
\end{aligned}
\]
This is \textcolor{red}{Backpropagation}!
\end{frame}

\begin{frame}
\begin{center}
\begin{tikzpicture}[thick,fill opacity=.3,draw opacity=3]
\draw[fill=red,rounded corners,] (1,-1) rectangle (3,1);
\draw[->] (4,0) -- (3,0);
\draw[->] (1,0.5) -- (0,1);
\draw[->] (1,-0.5) -- (0,-1); \pause
\node[right] at (4,0) {$\frac{\partial \mathcal{L}}{\partial y_j^{(l)}}$}; \pause
\node[] at (2,0) {$\frac{\partial y_j^{(l)}}{\partial y_j^{(l-1)}}$};
\pause
\node[left] at (0,1) {$\frac{\partial \mathcal{L}}{\partial y_j^{(l-1)}}$};\pause
\draw[->] (-1.5,1) -- (-3,1); \pause
\draw[fill=red,rounded corners,] (-5,0) rectangle (-3,2);
\pause
\node[] at (-4,1) {$\frac{\partial y_j^{(l-1)}}{\partial y_j^{(l-2)}}$};
\draw[->] (-5,1.5) -- (-6,2);
\draw[->] (-5,0.5) -- (-6,0);\pause 
\node[left] at (-6,2) {$\frac{\partial \mathcal{L}}{\partial y_j^{(l-2)}}$};
\end{tikzpicture}
\end{center}
\end{frame}

\begin{frame}
\frametitle{All types of Gradient Descent}
\begin{enumerate}
\item \textbf{Batch Gradient Descent:} uses entire dataset for one update.
\bigskip
\item \textbf{Minibatch Gradient Descent:} uses subsets of the dataset for one update.
\bigskip 
\item \textbf{Stochastic Gradient Descent (online):} uses single example for one update.
\end{enumerate}
\end{frame}

\begin{frame}
\frametitle{Downsides}
\begin{enumerate}
\item \textbf{Batch Gradient Descent:} sensitivity to saddle points and time-complexity, choosing a proper learning rate can be difficult.
\bigskip
\item \textbf{Minibatch Gradient Descent:} does not guarantee good convergence.
\bigskip 
\item \textbf{Stochastic Gradient Descent (online):} time-complexity, stuck at local minima.
\end{enumerate}
\end{frame}


\begin{frame}
\frametitle{Gradient}
Let $\{(\mathbf{X_1},y_1),\dots,(\mathbf{X_N},y_N)\}$ be the training set. then the update rules are as follow,
\begin{enumerate}
\item \textbf{Batch Gradient Descent:} 
\[
\mathbf{W}_{k+1}=\mathbf{W}_k-\alpha\nabla_{\mathbf{W}} \frac{1}{N}\sum_{i=1}^N\mathcal{L}(\mathbf{X_i},y_i;\mathbf{W}_k)
\]
\item \textbf{Minibatch Gradient Descent:}
 \[
\mathbf{W}_{k+1}=\mathbf{W}_k-\alpha \nabla_{\mathbf{W}}\frac{1}{\textbf{batch size}}\sum_{i\in \textbf{minibatch}}\mathcal{L}(\mathbf{X_i},y_i;\mathbf{W}_k)
\]
\item \textbf{Stochastic Gradient Descent (online):} \[
\mathbf{W}_{k+1}=\mathbf{W}_k-\alpha\nabla_{\mathbf{W}}\mathcal{L}(\mathbf{X_i},y_i;\mathbf{W}_k)
\]
\end{enumerate}
\end{frame}


\begin{frame}{}
  \centering \Huge
  \emph{Thank You}
\end{frame}

\end{document}

